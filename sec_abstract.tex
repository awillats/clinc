%identifying circuits in the brain is a major goal of neuroscience
A primary goal of neuroscience is identifying causal relationships between (regions / circuits) of the brain\todo{make this sound more neuro-legit},
%it's difficult.
a task that is challenging\todo{challenging $\to$ impossible?} when only observational (without intervention?) data is available despite being able to record from an increasing proportion of the brain [cite neuropixels].
% interventions have been used before -- crude and newer
Interventions that stimulate parts of the brain, such as lesioning and optogenetics\todo{more neuro here}, have been proposed to partially overcome this difficulty.
% but open-loop interventions insufficient
These open-loop interventions, however, do not fully eliminate confounding, limiting their ability to reveal casual structure in some circuits. This is especially problematic when circuits feature reciprocal connections, strong feedback at fast timescales, or when multiple regions are driven by the same upstream cause.
% closed loop + causal inference
In this paper, we show that interventions using closed-loop control can overcome these limitations.
% more specifics on paper
% - when is observational/open-loop sufficient / when is closed-loop required?
% - "what are the requirements for spatial and temporal degrees of freedom to meet our identification goals?"
% - how can aspects of an experiment be designed in order to strengthen (and make more data-efficient) our inferential power?
We provide a practical framework for applying closed-loop control\todo{add something about 3-node circuit?} to circuit identification problems and propose rules that predict when observational data or open-loop interventions are sufficient, and when closed-loop control is needed. Using a wide range of simulated circuits, we then characterize how key process and intervention parameters affect successful identification. This characterization builds towards an understanding of the limits of identification \todo{mention interventional budget?} and suggests methods for improving the next generation of circuit identification.
% broader impact
Our approach could complement existing technology developments and guide the next generation of (identification in neural circuits).


%further problem - we can execute closed-loop, but we don't know how to do so best for identification
In this work, we aim to provide a practical framework for applying closed-loop control to circuit identification problems. (The hope is this work) will (open up, further develop) a dialog between those designing and executing systems neuroscience experiments and computational neuroscientists about how to conduct identification experiments moving forward\todo{sounds condescending}. We aim to start answering questions such as
- what additional value does closed-loop control offer?
- in which circuits is closed-loop control necessary? in which circuits is open-loop control sufficient? What are the requirements for spatial and temporal degrees of freedom to meet our identification goals?
- what can be (said about) the connections identified from such experiments?
- How can aspects (parameters?) of an experiment be designed in order to strengthen (and make more data-efficient) our (hypothesis-testing power, the conclusions about circuits)

%results overview?
Overall, this paper summarizes the intersection of causal inference, neuroscience, and control theory, to highlight why and where intervention facilitates circuit identification. We then demonstrate how these principles apply to a simple case study of a 3-node circuit. Then we show a broad characterization of how several key process and intervention parameters affect successful identification. This characterization builds towards and understanding of the (limits of identification) and (begins to suggest) ways to (make the most of the next generation of circuit identification)

% broader impact - why
This is important because statements of causality are what we mean fundamentally when we talk about how the brain works. Statements of causality are also what we \textit{need} when seeking to develop new therapies to treat disorders of the brain.
