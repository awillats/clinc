\documentclass{article}
\usepackage[left=0.7in,right=1.1in]{geometry}
%\usepackage{default}
\usepackage{amsmath,amsthm,amssymb,amsfonts,bm,mathtools,braket,units,array,comment,cases}
\usepackage{fancyhdr,lastpage}
\usepackage{algorithm,algpseudocode}
\usepackage{makecell}
\usepackage[square,numbers]{natbib}
\usepackage{accents}
\usepackage{centernot}
\usepackage{filecontents}
\usepackage{lineno}
\usepackage{tikz}
\usepackage[colorinlistoftodos]{todonotes}

\usepackage[percent]{overpic}
\usepackage[caption=false, font=normalsize, labelfont=sf, textfont=sf]{subfig}




\newcommand{\todoblue}[2]{ \todo[#2,color=blue!20]{ \B{[note:]} #1 }}
%\newcommand{\todored}[2]{ \todo[#2,color=blue!20]{ \B{[note:]} #1 }}
\newcommand{\todocom}[2]{ \todo[#2,color=yellow!20]{ \B{[comment]} #1 }}
\newcommand{\mytodo}[1]{ \todoblue{#1}{noinline}}
\newcommand{\mycomment}[1]{ \todocom{#1}{noinline}}

\usetikzlibrary{matrix}
\usetikzlibrary{shapes.geometric}
\usetikzlibrary{arrows.meta}
\usetikzlibrary{positioning,shapes,arrows}
\usetikzlibrary{calc}
\usetikzlibrary{shapes.multipart}

\def\E{\mathbb{E}}
\def\G{\mathcal{G}}
\def\Pa{\mathrm{Pa}}
\def\R{\mathbb{R}}
\def\Reach{\mathcal{R}}%{\mathrm{Reach}}
\def\Control{\kappa_i}
\def\CL{\mathcal{C}}
\def\OL{\mathcal{O}}
\def\pluseqq{\mathrel{{+}{=}}}
\def\Undirect{\mathcal{U}}%{\mathrm{Undirect}}
\def\qhat{\widehat{q}}
\def\xhat{\widehat{x}}
\def\yhat{\widehat{y}}
\def\zhat{\widehat{z}}
\def\what{\widehat{w}}

\newtheorem{theorem}{Theorem}
\newtheorem{proposition}[theorem]{Proposition}
\newtheorem{lemma}[theorem]{Lemma}
\newtheorem{corollary}[theorem]{Corollary}
\newtheorem{assumption}[theorem]{Assumption}

\linenumbers

\title{Causal inference in neural circuits with open- and closed-loop control}
\author{Adam Willats and Matt O'Shaughnessy}
%\date{March 2020}

\bibliographystyle{plainnat}

\begin{document}

% Link to google slides figure plan
% https://docs.google.com/presentation/d/1w8q82VagGy9EmpgD1tJd2XS0f9lLHUMjUMG987uOb0Y/edit?usp=sharing

\maketitle

\begin{abstract}
    \input{0_abstract.tex}
\end{abstract}

\section{Introduction}
\label{sec:introduction}
\input{1_introduction.tex}

\section{Methods}
\label{sec:methods}
\input{2_methods.tex}

\section{Results}
\label{sec:results}

\input{3A_casestudy}
\input{3B_reachability}
\subsection{Characterizing identifiability in spiking neural networks}

\subsubsection{Methods - pipeline for inferring circuits with intervention}

\textbf{relevant properties of spiking circuits}
\textbf{within-cell properties}
\begin{itemize}
    \item membrane time constants
    \item internal noise
\end{itemize}

\textbf{properties of connctions}
\begin{itemize}
    \item connection strength (weights)
    \item synaptic delay
\end{itemize}

\textbf{properties across a circuit}
\begin{itemize}
    \item heterogeneity of (above properties) across neurons
    \item proportion of excitatory vs inhibitory connections
    \item \textit{(spiking vs lfp-like observations)}
\end{itemize}

\textbf{intervention parameters}
\begin{itemize}
    \item passive observation v.s. open-loop v.s. closed-loop control
    \item \% of the network observed
    \item \% of the network intervened
    \item number of samples recorded
    \item controller bandwidth
    \item \textit{(intervention location - stimulating driver versus "dead-end" nodes)}
    \item \textit{(impact of imperfect control)}
\end{itemize}


\textbf{Methods for extracting circuit estimates}
\textbf{Quantifying successful identification}
% accounting for / controlling for the influence of increased spiking on identifiability

\begin{itemize}
    \item false positives/negatives in identified links
    \item link-identification accuracy / F1 score
\end{itemize}


\textbf{Results - open-loop intervention facilitates identification with less data than passive observation}







\begin{figure}[ht]
	\centering
	 \begin{overpic}[width=.6\textwidth]{figures/figure_sketches/figure4a_sketch.png}
	  \end{overpic}
    \caption{\textbf{Text} more text}
    \label{fig:pipeline} %figure 1
 \end{figure}

\begin{figure}[ht]
	\centering
	 \begin{overpic}[width=.6\textwidth]{figures/figure_sketches/figure4b_sketch.png}
	  \end{overpic}
    \caption{\textbf{Text} more text}
    \label{fig:characterization} %figure 1
 \end{figure}


 \begin{figure}[ht]
 	\centering
 	 \begin{overpic}[width=.6\textwidth]{figures/figure_sketches/figure4c_sketch.png}
 	  \end{overpic}
     \caption{\textbf{Text} more text}
     \label{fig:res_summary} %figure 1
  \end{figure}


\section{Discussion}
\label{sec:discussion}
\subsection{Related work}
%move to intro?
% See: https://workflowy.com/s/toolboxes-software/a06de60HRO4CcHmm
% for a list of other software packages for Causal ID

\subsection{Challenges to identification, a spectrum of interventions}

\subsection{Limitations}

\subsection{Future Work}





\bibliography{closedloopcausal.bib}

\end{document}
